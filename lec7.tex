\documentclass[12pt,letter]{article}

%% \usepackage[fleqn]{amsmath}
\usepackage[margin=1in]{geometry}
\usepackage{amsmath,amsfonts,amsthm,bm}
\usepackage{breqn}
\usepackage{amsmath}
\usepackage{amssymb}
\usepackage{tikz}
\usepackage{algorithm2e}
\usepackage{siunitx}
\usepackage{graphicx}
\usepackage{subcaption}
%% \usepackage{datetime}
\usepackage{multirow}
\usepackage{multicol}
\usepackage{mathrsfs}
\usepackage{fancyhdr}
\usepackage{fancyvrb}
\usepackage{parskip} %turns off paragraph indent
\pagestyle{fancy}

\usepackage{xcolor}
\usepackage{mdframed}
  
\usetikzlibrary{arrows}

\DeclareMathOperator*{\argmin}{argmin}
\newcommand*{\argminl}{\argmin\limits}

\newcommand{\mathleft}{\@fleqntrue\@mathmargin0pt}
\newcommand{\R}{\mathbb{R}}
\newcommand{\Z}{\mathbb{Z}} 
\newcommand{\N}{\mathbb{N}}
\newcommand{\ppartial}[2]{\frac{\partial #1}{\partial #2}}
\newcommand{\p}{\partial}
\newcommand{\te}[1]{\text{#1 }}
\newcommand{\norm}[1]{\|#1\|}

\setcounter{MaxMatrixCols}{20}

% remove excess vertical space for align equations
\setlength{\abovedisplayskip}{0pt}
\setlength{\belowdisplayskip}{0pt}
\setlength{\abovedisplayshortskip}{0pt}
\setlength{\belowdisplayshortskip}{0pt}

\newtheorem{theorem}{Theorem}[section]
\newtheorem{corollary}{Corollary}[theorem]
\newtheorem{lemma}[theorem]{Lemma}

% \newtheorem{mdtheorem}{Theorem}
% \newenvironment{theorem}
% {\begin{mdframed}[
%     backgroundcolor=green!10,
%     topline=false,
%     rightline=false,
%     bottomline=false,
%     leftline=false
%     ]\begin{mdtheorem}}
%     {\end{mdtheorem}\end{mdframed}}

\begin {document}

\lhead{Notes - Numerical Optimization, 2020/01/22}

\section{2nd order conditions}

\begin{theorem}
  If $x^*$ is a local minimizer of $f:\R^n\to\R$ and $\nabla^2f(x)$ exists and is continuous in neighbourhood of $x^*$, then $\nabla f(x^*)=$ and $\nabla^2 f(x^*)$ is symmetric positive semi-definite.
\end{theorem}

\begin{proof}
  By contrdiction.\\
  Note from previous theorem $\nabla f(x^*)=0$ and $\nabla^2 f(x^*)$ is symmetric. So contrdiction must be $\nabla^2 f(x^*)$ is not PSD: $(\exists p) p^T\nabla^2 f(x^*)p <0$.\\
  
  By continuity of $\nabla^2 f(x)$ must have that there is an $\alpha>0$ such that:\\
  \[\phi(t)=p^T \nabla^2 f(x^*+t\alpha p)p <0$ for all $t\in(0,1)\].\\
  $\phi(t)=\nabla f(x^*+t\alpha p)^Tp<0$\\
  $f(x^*+\alpha p)=f(x^*)+\alpha \nabla f(x^*)^Tp + \frac{\alpha^2}{2}p^T \nabla^2 f(x^*+t \alphap)p$ for some $t\in(0,1)$\\
  $\nabla f(x^*)^T=0$, $\frac{\alpha^2}{2}p^T \nabla^2 f(x^*+t \alphap)p<0$\\
  We can always choose $\alpha$ small enough such that $x^*+\alpha p\in B(x^*,r)$
\end{proof}\\

\\\\


\begin{theorem}
  2nd order Sufficient condition for a minimizer.\\
  if $\nabla^2 f(x)$ exists and is continuous in a neighbourhood of $x^*$ and $\nabla f(x^*)=0$ and $\nabla^2 f(x^*)$ is symmetric positive definite, then $x^*$ is a local minimizer of $f$.
\end{theorem}
\begin{proof}
  Since $\nabla^2 f(x^*)$ is symmetric postiive definite and $\nabla^2 f(x)$ is continuous in a neighbourhood of $x^*$ there exists a $\delta>0$ s.t. $\nabla^2 f(x*+p)$ is SPD for all $\norm{p}<\delta$.\\
  $u^T \nabla^2 f(x^*+p) u > 0, \forall u \in \R^n, u\not=0$\\
  Choose any point $x\in B(x^*,\delta),x\not=x^*$\\
  Let $p=x-x^*$, $\norm{p}<\delta$\\
  So $f(x)=f(x^*+p)=f(x^*)+\nabla f(x^*)^Tp+\frac{1}{2}p^T\nabla^2 f(x^*+tp)p, t\in(0,1)$\\
  $\nabla f(x^*)^T=0$, $\frac{1}{2}p^T\nabla f(x^*)^Tp>0$, so:\\
  $f(x)>f(x^*+p)$
\end{proof}

\begin{theorem}
  Then if $f$ is convex, any loval minimer is a global minizer (does not imply a local minimizer is unique).
\end{theorem}
\begin{proof}
  By contradiction.\\
  Suppose $x^*$ is a local minimizer and there is some $y^*$ s.t. $f(y^*)<f(x^*)$
  \begin{align*}
    f(\alpha x^*)+(1-\alpha)y^*) & \leq \alpha f(x^*)+(1-\alpha)f(y^*)\\
                                 & < \alpha f(x^*) + (1-\alpha)f(x^*)=f(x^*), \alpha < 1\\
    z &= \alpha x^* + (1-\alpha)y^*\\
    f(z) &< f(x^*)
  \end{align*}
  For any $B(x^*,r),r>0$, we can choose $\alpha<1$ s.t. $z\in B(x^*,r)$
\end{proof}

\begin{theorem}
  If $f$ is convex, then any local minimizer is a global minimizer. In addition, if $f$ is continuously differentiable in a neighbourhood of $x^*$ and $\nabla f(x^*)=0$ then $x^*$ is a global minimizer of $f(x)$.
\end{theorem}

\end {document}
