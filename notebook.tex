\documentclass[8pt,letter]{article}

%% \usepackage[fleqn]{amsmath}
\usepackage[margin=1in]{geometry}
\usepackage{amsmath,amsfonts,amsthm,bm}
\usepackage{breqn}
\usepackage{amsmath}
\usepackage{amssymb}
\usepackage{tikz}
\usepackage[ruled,vlined,linesnumbered,lined,boxed,commentsnumbered]{algorithm2e}
\usepackage{siunitx}
\usepackage{graphicx}
\usepackage{subcaption}
%% \usepackage{datetime}
\usepackage{multirow}
\usepackage{multicol}
\usepackage{mathrsfs}
\usepackage{fancyhdr}
\usepackage{fancyvrb}
\usepackage{parskip} %turns off paragraph indent
\pagestyle{fancy}

\usepackage{xcolor}
\usepackage{mdframed}

\usetikzlibrary{arrows}

\DeclareMathOperator*{\argmin}{argmin}
\newcommand*{\argminl}{\argmin\limits}

\newcommand{\mathleft}{\@fleqntrue\@mathmargin0pt}
\newcommand{\R}{\mathbb{R}}
\newcommand{\Z}{\mathbb{Z}} 
\newcommand{\N}{\mathbb{N}}
\newcommand{\ppartial}[2]{\frac{\partial #1}{\partial #2}}
\newcommand{\p}{\partial}
\newcommand{\te}[1]{\text{#1 }}
\newcommand{\norm}[1]{\|#1\|}

\setcounter{MaxMatrixCols}{20}

% remove excess vertical space for align equations
\setlength{\abovedisplayskip}{0pt}
\setlength{\belowdisplayskip}{0pt}
\setlength{\abovedisplayshortskip}{0pt}
\setlength{\belowdisplayshortskip}{0pt}

\newtheorem{theorem}{Theorem}[section]
\newtheorem{corollary}{Corollary}[theorem]
\newtheorem{lemma}[theorem]{Lemma}

% \newtheorem{mdtheorem}{Theorem}
% \newenvironment{theorem}
% {\begin{mdframed}[
%   backgroundcolor=green!10,
%   topline=false,
%   rightline=false,
%   bottomline=false,
%   leftline=false
%   ]\begin{mdtheorem}}
%     {\end{mdtheorem}\end{mdframed}}

\begin {document}

\lhead{Notes - Numerical Optimization}

\begin{multicols*}{2}

  \section{General}

  line search conditions (1st and 2nd order):
  \begin{align}
    f_{k+1}^T & \leq f_k + c_1 \alpha_k \nabla f_k^T p_k\\
    \nabla f_{k+1}^T p_k & \geq c_2 \nabla f_k^T p_k\\
              &c_1,\alpha_k \in (0,1)\\
              &0<c_1<c_2<1\\
    \text{where}:& f_{k} = f(x_k)\\
              & f_{k+1} = f(x_k+\alpha_k p_k)
  \end{align}

  \begin{algorithm}[H]
    \SetKwInOut{Input}{$f, x, d, c_1, \alpha, \beta$}\SetKwInOut{Output}{$\alpha$}
    \Input{function, x, direction, gradient threshold, initial step length, contraction}
    \Output{step length}
    \While{$f(x + \alpha d) > f(x) + c_1 \alpha \nabla f(x)^T d$}{
      $\alpha \leftarrow \alpha * \beta$\\
    }
    return $\alpha$
    \caption{Line Search\label{Algo_LineSearch}}
  \end{algorithm}
  
  % Todo: add section on derivatives

  \section{Quasi Newton}

  \subsection{BFGS}
  properties: $O(n^2)$, self correcting, slightly more iterations than Newton Method, linear convergence order and superlinear rate of convergence
  
  secant equation:
  \begin{align*}
    B_{k+1}(x_{k+1}-x_k) & =\nabla f_{k+1} - \nabla f_k\\
    B_{k+1} s_k & = y_k\\
    s_k & = \alpha_k p_k\\
    y_k & = \nabla f_{k+1} - \nabla f_k\\
    B_{k+1} &:= \text{approx. Hessian}
  \end{align*}
  \begin{align*}
    B_{k+1} & \succ 0\\
    s_k^T B_{k+1} s_k & = s^T y_k > 0\\
  \end{align*}    
  \begin{proof}
    \begin{align*}
      y_k^T s_k & = (\nabla f_{k+1} - \nabla f_k)^T s_k\\
      \nabla f_{k+1}^T s_k & \geq c_2 \nabla f_k^T s_k\\
      (\nabla f_{k+1} - \nabla f_k)^T s_k & \geq c_2 \nabla f_k^T s_k - \nabla f_k^T s_k\\
      y_k^T s_k & \geq (c_2-1) \nabla f_k^T s_k\\
      c_2 < 1, s_k \text{ is a descent dir}& \implies s_k^T y_k > 0
    \end{align*}
    Curvature condition holds.
  \end{proof}

  constrain $B$ by solving:
  \begin{align*}
    \min_B & ||B-B_k||\\
    s.t. & \ B=B^T, Bs_k=y_k
  \end{align*}

  similarly, constrain $B$'s inverse, $H$ where it satisfy secant equation:
  \begin{align*}
    H_{k+1}y_k & = s_k\\
    \min_H & ||H-H_k||\\
    s.t. & \ H=H^T, Hy_k=s_k
  \end{align*}
  using weighted Frobenius norm:
  \begin{align*}
    ||A||_W & := ||W^{1/2}AW^{1/2}||_F\\
    ||X||_F & := (\sum_i \sum_j (X_{ij})^2)^{1/2}
  \end{align*}
  solved weight matrix $W$ satisfy $Ws_k=y_k$
  solution given by:
  \begin{align*}
    H_{k+1} & = (I-\rho_k s_k y_k^T)H_k(I-\rho_ky_ks_k^T) \\
            &+ \rho_k s_k^T s_k\\
    \rho_k & = \frac{1}{y_k^Ts_k}
  \end{align*}
  $W$ is the average Hessian $\bar{G}$:
  \begin{align*}
    \bar{G} = \int_0^1 \nabla^2 f(x_k+\tau \alpha_k p_k) d\tau
  \end{align*}
  
  initial $H_0$ can be chosen approximately (eg: finite differences, $I$)
  
  \begin{algorithm}[H]
    \SetKwInOut{Input}{$H_0,x_0,\epsilon>0$}\SetKwInOut{Output}{$x$}
    \Input{inverse Hessian approx., initial point, convergence tolerance}
    \Output{solution}
    $k \leftarrow$ 0\\
    $p_k \leftarrow -B^{-1} \nabla f(x_k) = -H\nabla f(x_k)$\\
    \While{$||\nabla f_k|| > \epsilon$}{
      $\alpha_k \leftarrow $ LineSearch(..)\\
      $x_{k+1} \leftarrow x_k + \alpha_k p_k$\\
      $s_k \leftarrow x_{k+1} - x_k$\\
      $y_k \leftarrow \nabla f_{k+1} - \nabla f_k$\\
      $\rho_k \leftarrow \frac{1}{y_k^Ts_k}$\\
      $H_{k+1} \leftarrow (I-\rho_k s_k y_k^T)H_k(I-\rho_ky_ks_k^T)\\
      + \rho_k s_k s_k^T$\\
      $k \leftarrow k+1$\\
    }
    return $x$
    \caption{BFGS Algorithm\label{Algo_BFGS}}
  \end{algorithm}
  
  using Sherman-Morrison-Woodbury formula to obtain Hessian update equation:
  \begin{align*}
    B_{k+1} & = B_k - \frac{B_ks_k s_k^T B_k}{s_k^T B_k s_k} + \frac{y_k y_k^T}{y_k^T s_k}
  \end{align*}
  proper line search is required so that BFGS algo captures curvature information\\
  
  inaccurate line search can be used to reduce computation cost
  \vfill\null
  
  % \pagebreak
  
  \section{Trust Region Methods}
  idea:
  \begin{itemize}
  \item models local behaviour of the objective function (eg: 2nd order Taylor series)
  \item set local region to explore, then simultaneously find direction and step size to take
  \item region size adaptively set using results from previous iterations
  \item step may fail due to inadequately set region, which need to be adjusted
  \item superlinear convergence when approximate model Hessian is equal to true Hessian
  \end{itemize}

  using 2nd order Taylor series model with symmetric matrix approximating Hessian
  \begin{align*}
    \min_{p \in \R^n} m_k(p) & = f_k + g_k^Tp + \frac{1}{2}p^TB_kp,\ st.\ ||p||\leq \Delta_k\\
    \Delta_k & :=\text{trust region radiius}\\
    g_k & =\nabla f(x_k)
  \end{align*}
  
  full step is ($p_k=-B_k^{-1}g_k)$ taken when $B\succ 0 $ and $||B_k^{-1} g_k|| \leq \Delta_k$

  evaluate goodness of model with actual function by:
  \begin{align*}
    &\rho_k = \frac{f(x_k)-f(x_k+p_k)}{m_k(0)-m_k(p_k)}\\
    &action \leftarrow
      \begin{cases}
        \text{expand trust region} &, \rho_k \approx 1\ (agreement)\\
        \text{shrink trust region} &, \rho_k  < 0 + \text{thresh}\\
        \text{keep trust region} &, o/w
      \end{cases}  
  \end{align*}

  \begin{algorithm}[H]
    $k \leftarrow$ 0\\
    \While{$||\nabla f_k|| > \epsilon$}{
      $p_k \leftarrow
      \argminl_{p} f_k + g_k^Tp + \frac{1}{2}p^TB_kp,\ st.\ ||p||\leq \Delta_k$\\
      $\rho_k = \frac{f(x_k)-f(x_k+p_k)}{m_k(0)-m_k(p_k)}$\\
      \uIf{$\rho_k<\gamma (:\frac{1}{4})$}{
        $\Delta_{k+1} \leftarrow \alpha (:\frac{1}{4}) \Delta_k$
      }
      \uElseIf{$\rho_k> \beta (:\frac{3}{4})\ and\ ||p_k|| == \Delta_k$}{
        $\Delta_{k+1} \leftarrow min(2 \Delta_k, \hat{\Delta})$
      }
      \Else{
        $\Delta_{k+1} \leftarrow \Delta_k$
      }
      
      \uIf{$\rho_k > \eta (: \in [0, \frac{1}{4}))$}{
        $x_{k+1} \leftarrow x_k + p_k$
      }
      \Else{
        $x_{k+1} \leftarrow x_k$
      }
    }
    \caption{Trust Region Algorithm\label{Algo_TrustRegion}}
  \end{algorithm}

  minimizer of the 2nd order Taylor series satisfy the following:
  \begin{align*}
    (B+\lambda I)p^*=-g\\
    \lambda(\Delta - ||p^*||)=0\\
    (B+\lambda I) \succeq 0
  \end{align*}
  
  solving 2nd order Taylor series using approx methods:
  \begin{itemize}
  \item dogleg
  \item 2-D subspace minimization
  \item conjugate gradient based
  \end{itemize}

  Cauchy Point:\\
  Use 1st order approx. of model and gradient descent to get get next iterate, bounded within trust region.\\
  
  \subsection{Dogleg method}
  if $B \succ 0$:\\
  $p^B=-B^{-1}g$\\
  $p^*=p^B$ if $\Delta \geq \|p^B\|$\\
  $p^U=\frac{-g^Tg}{g^TBg}g$ (intermediate point along direction of steepest descent)\\
  interpolate between $p^U$ and $p^B$:\\
  $\tilde{p}(\tau) =
  \begin{cases}
    \tau p^U &, \tau_\in [0,1]\\
    p^U + (\tau-1)(p^B-p^U) &, \tau \in [1,2]
  \end{cases}
  $\\
  $B \succ 0 \implies$ $\|\tilde{p}(\tau)\|$ increases wrt. $\tau$, $m(\tilde{p}(\tau))$ decreases wrt. $\tau$\\

  if $\|p^B\| \leq \Delta$: $p$ chosen at $p^B$\\
  else $p$ chosen at intersection of $\tilde{p}(\tau)$ and trust region boudnary by solving:\\
  $\|p^U+(\tau-1)(p^B-p^U)\|^2 = \|\Delta^2\|$

  $p_k^S=\argminl_p f_k+g_k^Tp, \|p\| \leq \Delta_k$\\
  $\tau_k = \argminl_{\tau\geq0} m_k(\tau p_k^S), \|\tau p_k^S\| \leq \Delta_k$\\
  $p_k^S=\frac{-\Delta_k g_k|}{\|g_k\|}$\\
  $p_k^C=\tau_k p_k^S$\\
  $p_k^C=-\tau_k \frac{\Deltak g_k}{\|g_k\|}$\\
  $\tau_k=\begin{cases}
    1 &, g_k^T B_k g_k \leq 0\\
    min(\frac{\|g_k|\^3}{\Delta_k g_k^T B_k g_k}, 1) &, o/w
  \end{cases}$
  
  \vfill\null
  % \pagebreak

  \subsection{Iterative Solution}
  Idea: solve subproblem $\min_{\|p\| \leq \Delta} m(p)$ by applying Newton's method to find $\lambda$ that matches trust region radius. This is slightly more accurate per step compared to Dogleg. Use $(B+\lambda I)p^*=-g$ to solve $\min_{\|p\| \leq \Delta} m(p)$ for $\lambda$.

  If $lambda=0$ and $(B+\lambda I)p^*=-g, \|p^*\| \leq \Delta$ and $(B+\lambda I) \succeq 0$: return $\lambda$\\
  Else: find $\lambda$ s.t. $(B+\labda I) \succeq 0$ and $\|p(\lambda)\| = \Delta, p(\lambda)=-(B+\lambda I)^{-1}g$. Solve and return $\lambda$.
  
  Solve $\|p(\lambda)\|-\Delta=0, \lambda > \lambda_1$ via Newton's method (root finding). Approx. this to nearly a linear problem for easy solving:\\

  \begin{algorithm}[H]
    \For{$l=0,1,..$}{
      solve $B+\lambda ^l I = R^T R$\\
      $R^TR p_l = -g$\\
      $R^T q_l = p_l$\\
      $\lambda^{l+1} \leftarrow \lambda^l + (\frac{\|p_l\|}{\|q_l\|})^2(\frac{\|p_l\|-\Delta)}{\Delta})$
      check $\lambda \geq \lambda_1$
    }
    \caption{Subproblem Algo\label{Algo_TrustRegionIterativeSubproblem}}
  \end{algorithm}

  
  \vfill\null
  \pagebreak
  
  \section{Conjugate Gradient}

  \subsection{linear method}
  Assuming unconstrained problem with strict convex quadratic objective function:
  \begin{align*}
    \frac{1}{2} x^TAx-b^Tx, A \succ 0, A^T=A
  \end{align*}
  
  $\nabla(\frac{1}{2} x^TAx-b^Tx) = Ax-b$, thus $\min_x x^TAx - b^Tx$ transformed to $Ax-b=0$.

  $x_{k+1} = x_k + \alpha_k p_k$, solve for $\alpha$:
  \begin{align*}
    &\frac{\partial}{\partial \alpha} (\frac{1}{2}(x_k + \alpha_k p_k)^T A (x_k + \alpha_k p_k) - b^T(x_k + \alpha_k p_k)) = 0\\
    &r_k = Ax-b\\
    &\alpha_k = \frac{ -p_k^T r_k}{p_k^T A p_k}\\
  \end{align*}
  \subsection{Conjugate Direction}
  Search directions linearly independent wrt. A.
  \begin{align*}
    (\forall i\not=j) p_i^T A p_j = 0
  \end{align*}
  Properties:
  \begin{itemize}
  \item Residual elimnated one direction at a time, resulting in max of n iterations.
  \item Optimal if Hessian is diagonal, if not can try preconditioning.
  \item Current residual is orthogonal to all previous search directions.
  \item Any set of conjugate directions can be used (eg: eigenvectors, Gram-Schmidt)
  \end{itemize}

  \vfill\null
  \columnbreak

  Expanding subspace minimizer:\\
  Using conjugate directions to generate sequence $\{x\}$, then:\\
  $r_k^T p_i = 0, \forall i < k$, $x_k$ is minimizer of $\frac{1}{2} x^TAx - b^Tx$ over $\{ x | x= x_0 + span\{p_0,...p_{k-1}\}$
  \begin{proof}
    \begin{align*}
      &\tilde{x} = x_0 + \sum_i \sigma_i p_i\\
      &\tilde{x} \text{ minimizes over }\{x_0 + span\{ p_0, ... p_{k-1}\}\} \iff r(\tilde{x})^Tp_i = 0\\
      &h(\sigma) = \phi(\tilde{x})\\
      &\phi(x) =\frac{1}{2}x^TAx-b^Tx\\
      &\text{h is also strictly convex quadratic,}\\
      &\text{with unique }\sigma^*\text{ satisfying:}\\
      &\frac{\partial h(\sigma^*)}{\partial \sigma_i} = 0, i=[0,k-1]\\
      &\frac{\partial h(\sigma^*)}{\partial \sigma_i} = \nabla \phi(\tilde{x})^T p_i = 0, i = [0,k-1]\\
      &\nabla \phi(x) = Ax-b=r\\
      &r(\tilde{x})^T p_i = 0, i=[0,k-1]\\
    \end{align*}  
  \end{proof}
  $p_i^T r_k=0, i=[0,k-1]$ via induction:
  \begin{proof}
    \begin{align*}
      &\text{base case}: x_1 = x_0 + \alpha_0 p_0 \text{ minimizes } \phi \text{ along } p_0\\
      &\implies r_1^T p_0 = 0\\
      &\text{case: }r_{k-1}^T p_i = 0, i=[0,k-2]:\\
      &r_k = r_{k-1} + \alpha_{k-1} A p_{k-1}\\
      &p_{k-1}^T r_k = p_{k-1}^T r_{k-1} + \alpha_{k-1} p_{k-1}^T  A p_{k-1}=0(\text{by construction)}\\
      &\text{A-conjugate} \implies p_{k-1}^T A p_{k-1}\\
      &\text{case: }\forall i=[0,k-2]: p_i^T r_k=0\\
      &p_i^T r_k = p_i^T r_{k-1} + \alpha_{k-1} p_i^T A p_{k-1}\\
      &p_i^T r_{k-1} = 0 (by induction hypothesis\\
      &\alpha_{k-1} p_i^T A p_{k-1} = 0 (conjugacy)\\
      &p_i^T r_k = 0, i=[0,k-1]
    \end{align*}
  \end{proof}

  \vfill\null
  \pagebreak
  
  \subsection{Conjugate Gradient Method}
  Idea:
  \begin{itemize}
  \item uses only previous search direction to compute current search direction
  \item $p_k$ set to linear combination of $-r_k$ and $p_{k-1}$
  \item impose $p_k^T A p_{k-1}=0$
  \end{itemize}
  \begin{align*}
    p_k &= -r_k + \beta_k p_{k-1}\\
    p_{k-1}^T A p_k &= -p_{k-1}^T A r_k + \beta p_{k-1}^T A p_{k-1}\\
    0 &= -p_{k-1}^T A r_k + \beta p_{k-1}^T A p_{k-1}\\
    \beta &= \frac{p_{k-1}^T A r_k}{p_{k-1}^T A p_{k-1}}\\
    p_0& = -(Ax_0-b)=-r_0
  \end{align*}
  \begin{algorithm}[H]
    $r_0 \leftarrow Ax_0-b$\\
    $p_0 = -r_0$\\
    \For{$k=[0,..n-1]$}{
      \uIf{$r_k==0$}{
        return $x_k$
      }
      \Else{
        $\alpha_k \leftarrow \frac{-r_k^Tp_k}{p_x^T A p_k}$\\
        $x_{k+1} \leftarrow x_k + \alpha_k p_k$\\
        $r_{k+1} \leftarrow Ax_{k+1} - b$\\
        $\beta_{k+1} \leftarrow \frac{p_k^T A r_{k+1}}{p_k^T A p_k}$\\
        $p_{k+1} \leftarrow -r_{k+1} + \beta_{k+1} p_k$
      }
    }
    \caption{Basic Conjugate Gradient Algorithm\label{Algo_CGBasic}}
  \end{algorithm}
  $p_$ and $r_k$ is within krylov subspace:\\
  $K(r_0;k) =span\{r_0, Ar_0, ..A^k r_0\}$\\
  if $r_k \not= 0$:\\
  $r_k^T r_i = 0, i=[0,k-1]$\\
  $span\{r_0,..,r_k\} = span\{r_0,Ar_0,..,A^k r_0\}$\\
  $span\{p_0,..,p+k\}=span\{r_0,Ar_0,..,A^k r_0\}$\\
  $p_k^T A p_i=0, i=[0,k-1]$\\
  then, $\{x_k\} \rightarrow x^*$ in at most n steps.\\

  \vfill\null
  \columnbreak
  
  Simplification:
  \begin{align*}
    &p_{k+1} \leftarrow -r_{k+1} + \beta_{k+1} p_k\\
    &\alpha_k \leftarrow \frac{-r_k^Tp_k}{p_k^T A p_k}\\
    &\alpha_k \leftarrow \frac{-r_k^T(-r_{k} + \beta_{k} p_{k-1})}{p_k^T A p_k}\\
    &(\forall i=[0,k-1]) r_k^T p_i=0 \implies \beta_k r_k^T p_{k-1}=0\\
    &\alpha_k \leftarrow \frac{r_k^T r_k}{p_k^T A p_k} \ \text{(simplified)}
  \end{align*}
  \begin{align*}
    &r_{k+1}=r_k+\alpha_k A p_k\\
    &A p_k = \frac{r_{k+1}-r_k}{\alpha_k}\\
    &\beta = \frac{p_k^T A r_{k+1}}{p_k^T A p_k}\\
    &p_k^T A p_k = p_k^T \frac{r_{k+1}-r_k}{\alpha_k}= \frac{-p_k^T r_k}{\alpha} (\text{conjugacy})\\
    &p_k^T A p_k = -\frac{(- r_k + \beta_k p_{k-1})^T r_k}{\alpha}=\frac{r_k^T r_k}{\alpha} (\text{conjugacy})\\
    &p_k^T A r_{k+1} = r_{k+1}^T A p_k\\
    &p_k^T A r_{k+1} = r_{k+1}^T \frac{r_{k+1}-r_k}{\alpha_k}\\
    &r_k \in span\{p_k,p_{k-1}\}\ and\ r_{k+1}^T p_i = 0, i=[0,k] \implies\\
    &p_k^T A r_{k+1} = \frac{r_{k+1}^T r_{k+1}}{\alpha_k}\\
    &\beta_{k+1} \leftarrow \frac{r_{k+1}^T r_{k+1}}{r_k^T r_k} \ \text{(simplified)}
  \end{align*}

  \subsection{Nonlinear Method}
  Minimize general convex function or nonlinear function. Variants: FR, PR.

  \subsubsection{FR (Fletcher Reaves)}
  Modify linear CG by:
  \begin{itemize}
  \item replace residual by gradient of objective, $r_k \rightarrow \nabla f_k$
  \item replace $\alpha_k$ computation by a linear search to find approx. minimum along search direction
  \end{itemize}

  Equivalent to linear CG if objective is strongly convex quadratic.\\

  Linear search for $\alpha_k$ with strong Wolfe condition to ensure $p_k$'s are descent directions wrt. objective function.\\
  
  \subsubsection{PR}
  Replace $\beta_{k+1}$ computation in FR with:
  \begin{align*}
    \beta_{k+1} \leftarrow \frac{\nabla f_{k+1}^T (\nabla f_{k+1} - \nabla f_k)}{\nabla f_k^T \nabla f_k}
  \end{align*}
  
  \vfill\null
  \columnbreak
  
  \section{Proximal Algorithm}
  Idea:
  \begin{itemize}
  \item reliance on easy to evaluate proximal operators
  \item separability allows parallel evaluation
  \item generalization of projection based algorithms
  \end{itemize}

  \begin{align*}
    prox_{\lambda f}(v) = \argminl_{x} f(x) + \frac{1}{2 \lambda} \|x-v\|_2^2
  \end{align*}

  Resolvent of subdifferential operator:
  \begin{align*}
    &z=prox_{\lambda f}(x) \implies z \in (I+\lambda \partial f)^{-1}(x)\\
    &(I+\lambda \partial f)^{-1} :=\text{resolvent of operator }\partial f
  \end{align*}
  
  \subsection{Proximal Gradient Method}
  Solve $\min_{x} g(x)+f(x)$, where $f,g$ are closed, convex functions and $f$ differentiable
  \begin{align*}
    &x^* = prox_{\lambda^k g}(x^k-\lambda^k \nabla f(x^k)) \\
    &=\argminl_{x} g(x) + \frac{1}{2 \lambda^k} \|x - (x^k - \lambda^k \nabla f(x^k))\|_2^2\\
    &\text{tradeoff between g and and gradient step}\\
    &g=I_C(x) \implies \text{projected gradient step}\\
    &g=0 \implies \text{gradient descent}\\
    &f=0 \implies \text{proximal minimization}
  \end{align*}

  Relation to Pixed Point:\\
  $x^*$ is a fixed point solution of $\min_{x} g(x)+f(x)$ iff $0 \in \nabla f(x^*) + \partial g(x^*)$ iff $x^* = (I+\lambda \partial g)^{-1}(I-\lambda \nabla f)(x^*)$\\
  
  Forward Euler, Backward Euler stepping is same as the proximal gradient iteration, $prox_{\lambda^k g}(x^k-\lambda^k \nabla f(x^k))$\\

  \subsection{Accelerated Proximal Gradient Method}
  Introduce extrapolation:
  \begin{align*}
    &y^{k+1} = x^k + w^k(x^k-x^{k-1})\\
    &x^{k+1} = prox_{\lambda^k g}(y^{k+1}-\lambda^k \nabla f(y^{k+1}))\\
    &w^k \in [0,1)
  \end{align*}

  Example: $w^k = \frac{k}{k+3}, w^0 = 0, \lambda^k \in (0,1/L], L:= \text{Lipschitz constant of }\nabla f,\ or\ \lambda^k$ found via line search.
  Line search for $\lambda^k$ (Beck and Teboulle):

  \begin{algorithm}[H]
    $\hat{f}(x,y):=f(y) + \nabla f(y)^T(x-y) + \frac{1}{2\lambda}\|x-y\|^2_2$\\
    \While{True}{
      $z = prox_{\lambda g}(y^k-\lambda \nabla f(y^k))$\\
      \uIf{$f(x) \leq \hat{f}(z,y^k)$}{
        break
      }
      $\lambda = \beta \lambda$
    }
    return $\lambda^k := \lambda, x^{k+1} := z$\\
    \caption{Proximal Gradient Algorithm\label{Algo_ProximalGradient}}
  \end{algorithm}

  \subsection{Types of Proximal Operators}
  \begin{itemize}
  \item quadratic functions
    \begin{align*}
      f&=\frac{1}{2} \|.\|_x^2 \implies prox_{\lambda f}(v) = (\frac{1}{1+\lambda})v\\
      f&=\frac{1}{2}x^T A x + b^T x + c, A \in S_+^n \implies\\
      &prox_{\lambda f}(v)=(I+\lambda)^{-1}(v-\lambda b)
    \end{align*}
  \item unconstrained problem: use gradient methods such as Newton, Quasi-Newton
  \item constrained: use projected subgradient for non-smooth, projected gradient or interior method for smooth
  \item separable function: if scalar, may be solved analytically, eg:
    L1 norm separable to:
    \begin{align*}
      f(x) = |x| \implies prox_{\lambda f}(v) =
      \begin{cases}
        v-\lambda ,& v \geq \lambda\\
        0,& |v| \leq \lambda\\
        v+\lambda, & v \leq -\lambda
      \end{cases}
    \end{align*}
    \begin{align*}
      f(x) = -log(x) \implies prox_{\lambda f}(v) = \frac{v+\sqrt{v^2+4\lambda}}{2}
    \end{align*}
  \item general scalar function
    \begin{itemize}
    \item localization: using a subgradient oracle and bisection algorithm
    \item twice continuously differentiable: guarded Newton method
    \end{itemize}
  \item polyhedra constraint, quadratic objective: solve as QP problem
    \begin{itemize}
    \item duality to reduce number of variables to solve if possible
    \item gram matrix caching
    \item affine constraint($Ax=b$): use pseudo-inverse, $A^{+}$:
      \begin{align*}
        \Pi_C(v) &= v - A^{+}(Av-b)\\
                 &A \in R^{m \times n}, m < n \implies A^{+} = A^T(AA^T)^{-1}\\
                 &A \in R^{m \times n}, m > n \implies A^{+} = (A^TA)^{-1}A^T
      \end{align*}
    \item hyperplane constaint($a^Tx=b$):
      \begin{align*}
        \Pi_C(v) = v + (\frac{b-a^Tb}{\|a\|_2^2})a
      \end{align*}
    \item halfspace
      \begin{align*}
        \Pi_C(v) = v - \frac{max(a^T v -b,0)}{\|a\|_2^2}a
      \end{align*}
    \item box($l\leq x \leq u$)
      \begin{align*}
        \Pi_C(v)_k = min(max(v_k, l_k), u_k)
      \end{align*}
    \item probability simplex($1^Tx= 1, x \geq 0$)\\
      bisection algo on $\nu$:
      \begin{align*}
        \Pi_C(v) = (v- \nu 1)_+\\
        \text{intial }[l_k,u_k] = [\max_i v_i-1, \max_i vi]
      \end{align*}
      analytically solve when bounded in between 2 adjacent v_i's      
    \end{itemize}
  \item cones ($\kappa$: proper cone)\\
    problem of the form:
    \begin{align*}
      min_x \|x-v\|_2^2\\
      s.t.: x\in \kappa
    \end{align*}
    \begin{align*}
      x \in \kappa\\
      v=x-\lambda\\
      \lambda\in\kappa^*\\
      \lambda^Tx=0
    \end{align*}
    \begin{itemize}
    \item cone $C=\R_+^n$
      \begin{align*}
        \Pi_C(v)=v_+
      \end{align*}
    \item 2nd order cone $C=\{(x,t)\in \R^{n+1}: \|x\|_2 \leq t\}$
      \begin{align*}
        \Pi_C(v,s) = \begin{cases}
          0,& \|v\|_2 \leq -s\\
          (v,s),& \|v\|_2 \leq s\\
          \frac{1}{2}(1+\frac{s}{\|v\|_2})(v,\|v\|_2),& \|v\|_2 \geq |s|
        \end{cases}
      \end{align*}
    \item PSD cone $S_+^n$
      \begin{align*}
        \Pi_C(V)&=\sum_i (\lambda_i)_+ u_i u_i^T\\
        V&=\sum_i \lambda_i u_i u_i^T\ (eigendecomp)
      \end{align*}
    \item exponential cone\\
      Todo
    \end{itemize}
  \item pointwise supremum
    \begin{itemize}
    \item max function
    \item support function
    \end{itemize}
  \item norms
    \begin{itemize}
    \item L2
    \item L1
    \item L-$\inf$
    \item elastic net
    \item sum of norms
    \item matrix norm
    \end{itemize}
  \item sublevel set
  \item epigraph
  \item matrix functions
    Todo
  \end{itemize}

  \section{Subgradient Method}
  todo
\end {document}
